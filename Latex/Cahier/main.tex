\documentclass{article}
\usepackage[utf8]{inputenc}

\usepackage{fullpage}
\usepackage{graphics}
\usepackage{hhline}
\usepackage{xcolor}
\usepackage{subcaption}
\usepackage{lipsum}


\title{Cahier des charges - Cook'in VR}
\author{4D Generates\\Gustave Hervé, Léo Sambrook, Arthur Hamard et Briac Guellec}
\date{Janvier 2022}

\begin{document}

\maketitle
\newpage

\renewcommand*\contentsname{Table des matières}
\tableofcontents
\newpage

\section{Introduction}

    Ce cahier des charges contient tout ce qu'il y a a savoir concernant au projet de S2 intitulé Cook'in VR.

\subsection{Qu’est-ce que Cook’in VR ?}

	Le projet Cook’in VR consiste en un jeu vidéo multijoueur axé sur la coopération ayant pour thème principal la cuisine. Le but du jeu consiste à remplir un maximum de commandes dans un temps imparti, le tout en faisant le moins d'erreurs possible. La particularité de ce jeu est que celui-ci se joue exclusivement en réalité virtuelle, ce qui explique l’inclusion de l’acronyme VR pour Virtual Reality dans le titre du jeu.

\subsection{Pourquoi un jeu de cuisine ?}

	Avant même de se concrétiser en un jeu centré sur la cuisine, l’objectif de 4D Generates était de faire un jeu qui serait intégralement jouable en réalité virtuelle. Nous venions en effet de nous procurer un casque Oculus Quest 2 pendant le S1 et nous étions impressionnés par le fonctionnement de la technologie et l’immersion procurée. Il nous a alors semblé évident de faire un jeu tirant parti des possibilités qui nous sont offertes par ces casques de réalité virtuelle.\\

En ce qui concerne le jeu en lui-même et sa thématique, l'idée derrière Cook’in VR vient à l'origine d'une blague consistant à faire un simulateur de restaurant type SpeedBurger en réalité virtuelle. Ce concept bien que paraissant absurde au premier abord, nous a semblé finalement très intéressant sur de nombreux points. En effet, il permet d’exploiter non seulement la réalité virtuelle par le biais de l’interaction, mais aussi le multijoueur coopératif de façon pertinente. Les cuisines étant justement, (en effet), affaire de coordination et de travail d’équipe, nous avons alors vu un potentiel certain pour un jeu coopératif se déroulant au sein de cette thématique.


\end{document}
